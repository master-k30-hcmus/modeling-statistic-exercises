\documentclass[a4paper]{article}
\usepackage[utf8]{vietnam}
\usepackage{scrextend}
\changefontsizes{13pt}
\usepackage{xcolor}
\usepackage{titlesec}
\usepackage{mdframed}
\usepackage{amsmath}
\usepackage{array}
\usepackage[amsmath,standard,thmmarks]{ntheorem}
\usepackage{amssymb}
\usepackage{exscale}
\usepackage{amsfonts}
\usepackage{eucal}
\usepackage{enumerate}
\usepackage{commath}
\usepackage{graphicx}
\usepackage{tcolorbox}
\usepackage{url}
\usepackage[unicode]{hyperref}
\newmdenv[linecolor=black,skipabove=\topsep,skipbelow=\topsep,
leftmargin=-5pt,rightmargin=-5pt,
innerleftmargin=5pt,innerrightmargin=5pt]{mybox}
\setlength{\parindent}{0pt}
 \usepackage[left=2cm,right=2cm,top=2.5cm,bottom=2.5cm]{geometry}
\renewcommand{\baselinestretch}{1.5}
\newcommand{\heva}[1]{\left\{ 
\begin{aligned}#1\end{aligned}\right.}

\theoremstyle{nonumberplain} 
\theoremheaderfont{\itseries\slshape} 
\theorembodyfont{\normalfont}
\theoremsymbol{\ensuremath{_\blacksquare}} 
\renewtheorem{proof}{Chứng minh:}
\newcommand{\chm}[1]{\begin{proof}#1\end{proof}}

\usepackage{fancyhdr}
\pagestyle{fancy}
\fancyhf{}
\lhead{BTL2 - Mô hình hóa thống kê}
\cfoot{\thepage}
\rhead{Nhóm 4}
\title{BÀI TẬP LẦN 2 - MHHTK}

\allowdisplaybreaks

\author{NHÓM 4}

\date{\today}%

\begin{document}

\begin{titlepage}
\thispagestyle{empty}
\begin{center}
\textbf{\large{TRƯỜNG ĐẠI HỌC KHOA HỌC TỰ NHIÊN TP.HCM}\\
CAO HỌC KHÓA 30}\\
---------------*---------------
\end{center}
\vspace{0.3cm}
\begin{center}
\includegraphics[scale=0.8]{Logo-Math-CS.png} 
\end{center}
\vspace{0.7cm}
\begin{center}
\textbf{\Large{\textcolor[rgb]{1.0,0.0,0.0}{Bài tập lần 2}}}\\
\vspace{0.5cm}
\textbf{\Large{\textcolor[rgb]{1.0,0.0,0.0}{MÔ HÌNH HÓA THỐNG KÊ}}}\\
\vspace*{4cm}
$\begin{array}{rl}
\text{\large{Giảng viên hướng dẫn:}} &\text{\large{\bf TS. Nguyễn Thị Mộng Ngọc}}  \\
\vspace*{1cm}
\text{\large{Nhóm thực hiện:}}     & \text{\large{\textbf{Nhóm 4}}}
\end{array}$\\
\vspace{4cm}
\normalsize{TP. Hồ Chí Minh $-$ Tháng 01, 2021}
\end{center}
\end{titlepage}

\newpage
\thispagestyle{empty}
\begin{center}
\textbf{\large{BẢNG PHÂN CÔNG CÔNG VIỆC}\\}
\vspace{1cm}
\begin{tabular}{|m{4.5cm}||m{8cm}|m{3.5cm}|} 
\hline
\textbf{Thành viên} & \centering{\textbf{Công việc}} & \textbf{Mã số học viên}\\
\hline
1. Đặng Khánh Thi & $-$ & \\
& $-$  & \hspace{0.75cm}20C29038 \\
& $-$ & \\
\hline
2. Đinh Thị Nữ  & $-$ & \\
& $-$ & \hspace{0.75cm}20C29013\\
& $-$ & \\
\hline 
3. Lý Phi Long & $-$ & \\
& $-$ & \hspace{0.75cm}20C29028\\
& $-$ & \\
\hline
4. Phan Thị Thùy An & & \\
\hspace{1cm}(Nhóm trưởng) & $-$  & \hspace{0.75cm}20C29002 \\
& $-$  & \\
\hline
\end{tabular}
\end{center}

\newpage
\section*{BÀI 1}
- $X_1$: áp lực công việc\\
- $X_2$: kỹ năng quản lý\\
- $X_3$: mức độ hài lòng với chức vụ của mình\\
- $Y$: mức độ lo lắng (biến phụ thuộc)\\
Bảng ANOVA:
\begin{center}
\includegraphics[scale =0.5]{anova1.PNG} 
\end{center}
\subsubsection*{1. Tính tổng bình phương hồi quy trên $X_1, X_2$ và $X_3$?}
$SSR = SSR_{X_1} + SSR_{X_2|X_1} + SSR_{X_3|X_1,X_2} = 981.326 + 190.232 + 129.431 = 1299.989$
\subsubsection*{2. Xác định tỷ lệ phần trăm sự biến thiên của mức độ lo lắng được giải thích bởi các biến độc lập.}

\[R^2 = \dfrac{SSR}{SST} = \dfrac{1299.989}{1743.281} = 0.7462876\]
Sự biến thiên của mức độ lo lắng được giải thích bởi các biến độc lập có tỉ lệ phần trăm là 74\%.

\subsubsection*{3. Có thể kết luận rằng trong tất cả ba biến giải thích đều có ảnh hưởng đáng kể đến mức độ lo lắng hay không? Chỉ rõ kiểm định nào được dùng.}

Đặt giả thuyết kiểm định:
\[\begin{cases}
	H_0: \beta_1 = \beta_2 = \beta_3 = 0\\
	H_1: \beta_1 \neq 0 \vee \beta_2 \neq 0 \vee \beta_3 \neq 0
\end{cases}\]

Với giả thuyết trên, không thể kết luận cả ba biến đều ảnh hưởng đáng kể đến mức độ lo lắng, mà chỉ có thể kết luận rằng tồn tại ít nhất một biến có ảnh hưởng đến mức độ lo lắng nếu giả thiết $H_0$ sai.

Ta tính được kiểm định Fisher cho quan trắc:
\[F_{obs} = \dfrac{SSR/(p-1)}{SSE/(n-p)} = 17.64882\]

Với mức ý nghĩa $\alpha = 0.05$, tra bảng thống kê Fisher ta được:
\[F_{1-\alpha}(p-1,n-p) = F_{0.95}(3,18) = 3.609\]

Vì $F_{obs} > F_{0.95}(3,18)$ nên ta bác bỏ $H_0$ với mức ý nghĩa 5\%.

Vậy tồn tại \textbf{ít nhất} một biến có ảnh hưởng đến mức độ lo lắng.

\subsubsection*{4. Nếu chúng ta chỉ xét biến giải thích $X_1$, hãy lập bảng ANOVA ?}

Khi chỉ xét $X_1$, mô hình hồi quy trở thành:
\[Y = \beta_0 + \beta_1X_1\]

Vậy tổng sai số của biến giải thích $X_1$ là:
\[SSE_{X_1} = SST - SSR_{X_1} = 761.955\]

Với số mẫu $n=22$, ta lập được bảng ANOVA với biến giải thích $X_1$ như sau:

\begin{center}
	\begin{tabular}{|c|c|c|c|c|}
		\hline
		Biến thiên & SS & DF & MS & Fisher\\
		\hline
		$R_{X_1}$ & $SSR_{X_1} = 981.326$ & 1 &$SSR_{X_1}/1 = 981.326$ & \\
		\hline
		$E_{X_1}$ & $SSE_{X_1} = 761.955$ & $n-2 = 20$ & $SSE_{X_1}/20 = 38.09775$ & $MSR_{X_1}/MSE_{X_1} = 25.75811$\\
		\hline
		Total & 1743.281 & $n-1 = 21$ & & \\
		\hline
	\end{tabular}
\end{center}

\subsubsection*{5. Kiểm định giả thuyết sau với mức ý nghĩa 5\%}

\textbf{(a)} $H_0 : \beta_1 = 0$ \textbf{cho mô hình } $Y = \beta_0 + \beta_1 X_1 + \epsilon $ 

Đặt giả thuyết kiểm định:
\[\begin{cases}
	H_0 : \beta_1 = 0\\
	H_1 : \beta_1 \ne 0
\end{cases}\]
Thống kê của kiểm định: 

$$F =  \displaystyle \frac{MSR}{MSE} \sim F_{(1,20)} \text{ khi } H_0 \text{ đúng},$$ 
với $F_{(1,20)}$ là phân phối Fisher có bậc tự do 1 và 20.

Dựa vào bảng ANOVA ở câu 4, ta tính được giá trị thống kê:
$$F_{obs} = \displaystyle \frac{981.326}{38.09775} = 25.7581$$

Với mức ý nghĩa $\alpha = 0.05$, tra bảng thống kê Fisher ta được:
$$F_{1-\alpha}(1,n-2) = F_{0.95}(1,20) = 4.3512$$

Vì $F_{obs}>F_{0.95}(1,20)$ nên ta bác bỏ $H_0$ với mức ý nghĩa 5\%.\\


\textbf{(b)} $H_0 : \beta_2 = 0$ \textbf{cho mô hình } $Y = \beta_0 + \beta_1 X_1 + \beta_2 X_2 + \epsilon $

Đặt giả thuyết kiểm định:
\[\begin{cases}
	H_0 : \beta_2 = 0 \text{ hay } Y = \beta_0 + \beta_1 X_1 + \epsilon \\
	H_1 : \beta_2 \ne 0 \text{ hay } Y = \beta_0 + \beta_1 X_1 + \beta_2 X_2 + \epsilon
\end{cases}\]
Thực hiện kiểm định Fisher từng phần, ta có thống kê của kiểm định: 

$$F = \displaystyle \frac{\left [ SSE (H_0) - SSE(H_1) \right ] / r}{SSE(H_1)/(n-p)} \sim F_{(r,n-p)} \text{ khi } H_0 \text{ đúng},$$
trong đó $r = 1, n = 22, p = 3$.

Trước tiên, ta cần tính $SSE (H_0)$ và $SSE(H_1)$:
$$SSE(H_0) = SST - SSR_{X_1} = 761.955 \text{, (đặt là } SSE_{X_1})$$
$$SSE(H_1) = SST - SSR_{X_1} - SSR_{X_2|X_1} = 571.723 \text{, (đặt là }  SSE_{X_1,X_2})$$
Giá trị thống kê $$F_{obs} = 6.3219$$
Với mức ý nghĩa $\alpha = 0.05$, tra bảng thống kê Fisher ta được:
$$F_{1-\alpha}(r,n-p) = F_{0.95}(1,19) = 4.3807$$

Vì $F_{obs}>F_{0.95}(1,19)$ nên ta bác bỏ $H_0$ với mức ý nghĩa 5\%.\\


\textbf{(c)} $H_0 : \beta_3 = 0$ \textbf{cho mô hình } $Y = \beta_0 + \beta_1 X_1 + \beta_2 X_2 + \beta_3 X_3 + \epsilon $

Đặt giả thuyết kiểm định:
\[\begin{cases}
	H_0 : \beta_3 = 0 \text{ hay } Y = \beta_0 + \beta_1 X_1 + \beta_2 X_2 + \epsilon \\
	H_1 : \beta_3 \ne 0 \text{ hay } Y = \beta_0 + \beta_1 X_1 + \beta_2 X_2 + \beta_3 X_3 + \epsilon 
\end{cases}\]
Thực hiện kiểm định Fisher từng phần, ta có thống kê của kiểm định: 

$$F = \displaystyle \frac{\left [ SSE (H_0) - SSE(H_1) \right ] / r}{SSE(H_1)/(n-p)}  \sim F_{(r,n-p)} \text{ khi } H_0 \text{ đúng},$$
trong đó $r = 1, n = 22, p = 4$.

Trước tiên, ta cần tính $SSE (H_0)$ và $SSE(H_1)$:
$$SSE(H_0) = SST - SSR_{X_1} - SSR_{X_2|X_1} = 571.723 \text{, (đặt là }  SSE_{X_1,X_2}) $$
$$SSE(H_1) = SST - SSR_{X_1} - SSR_{X_2|X_1} - SSR_{X_3|X_1,X_2} = 442.292 \text{, (đặt là }  SSE_{X_1,X_2,X_3})$$
Giá trị thống kê $$F_{obs} = 5.2675$$
Với mức ý nghĩa $\alpha = 0.05$, tra bảng thống kê Fisher ta được:
$$F_{1-\alpha}(r,n-p) = F_{0.95}(1,18) = 4.4138$$

Vì $F_{obs}>F_{0.95}(1,18)$ nên ta bác bỏ $H_0$ với mức ý nghĩa 5\%.

\subsubsection*{6. Xác định hệ số xác định cho mỗi mô hình trong câu 5.}

\textbf{Mô hình 1:} $Y = \beta_0 + \beta_1 X_1 + \epsilon $ có hệ số xác định là
$$R^2_1 = 1 - \displaystyle \frac{SSE_{X_1}}{SST} = 0.5629$$


\textbf{Mô hình 2:} $Y = \beta_0 + \beta_1 X_1 + \beta_2 X_2 + \epsilon $ có hệ số xác định là
$$R^2_2 = 1 - \displaystyle \frac{SSE_{X_1,X_2}}{SST} = 0.6720$$
\textbf{Mô hình 3:} $Y = \beta_0 + \beta_1 X_1 + \beta_2 X_2 + \beta_3 X_3 + \epsilon $ có hệ số xác định là
$$R^2_3 = 1 - \displaystyle \frac{SSE_{X_1,X_2,X_3}}{SST} = 0.7463$$



\subsubsection*{7. Trong các mô hình trên, mô hình nào thích hợp nhất để giải thích sự biến động mức độ lo lắng của các giám đốc ?}
Để so sánh độ thích hợp của các mô hình, ta cần so sánh các hệ số xác định hiệu chỉnh theo công thức:
$$R^2_{adj} = 1 - \displaystyle \frac{SSE/(n-p)}{SST/(n-1)} \text{ với } n = 22$$
\textbf{Mô hình 1:} với $p=2$ ta có $$R^2_{adj_1}  = 1 - \displaystyle \frac{SSE_{X_1}/20}{SST/21} = 0.5411$$ 

\textbf{Mô hình 2:} với $p=3$ ta có $$R^2_{adj_2} = 1 - \displaystyle \frac{SSE_{X_1,X_2}/19}{SST/21} = 0.6375$$

\textbf{Mô hình 3:} với $p=4$ ta có $$R^2_{adj_3} = 1 - \displaystyle \frac{SSE_{X_1,X_2,X_3}/18}{SST/21} = 0.7040$$

Dựa vào các giá trị $R^2$ hiệu chỉnh vừa tính, có thể kết luận \textbf{mô hình 3} là mô hình thích hợp nhất để giải thích sự biến động mức độ lo lắng của các giám đốc.
\newpage
\section*{BÀI 2}

\begin{figure}[h!]
	\centering
	\includegraphics[width=0.6\linewidth]{bai-2-data}
	\caption{Bảng số liệu bài 2}
	\label{fig:bai-2-data}
\end{figure}

\begin{enumerate}[-]
	\item $Y$: mức độ bền dẻo của nhựa
	\item $X_1$: độ dày của vật liệu
	\item $X_2$: mật độ của vật liệu
\end{enumerate}

\subsubsection*{1. Tìm 2 phương trình đường thẳng hồi quy và 1 phương trình siêu phẳng (nếu có) ?}

Ta xây dựng các mô hình hồi quy như sau:

\textbf{Mô hình 1:} $Y= \beta_0 + \beta_1 X_1 + \epsilon$\\
Mô hình đường thẳng hồi quy tương ứng: $Y= \hat{\beta_0} + \hat{\beta_1} X_1$

\begin{figure}[h!]
	\centering
	\includegraphics[scale =0.9]{bai2_1i.PNG}
	\caption{Tham số mô hình 1}
	\label{ex2:model:11}
\end{figure}

Dựa vào kết quả của phần mềm R ở hình \ref{ex2:model:11}, ta có $\hat{\beta_0} = 3.523$ và $\hat{\beta_1} = 6.036$, do đó ta có phương trình đường thẳng hồi quy theo độ dày của vật liệu ($X_1$) là:
\[Y = \hat{\beta_0} + \hat{\beta_1} X_1 = 3.523 + 6.036 X_1\]


\textbf{Mô hình 2:} $Y= \beta_0 + \beta_2 X_2 + \epsilon$\\
Mô hình đường thẳng hồi quy tương ứng: $Y= \hat{\beta_0} + \hat{\beta_2} X_2$

\begin{figure}[h!]
	\centering
	\includegraphics[scale =0.9]{bai2_1ii.PNG} 
	\caption{Tham số mô hình 2}
	\label{ex2:model:12}
\end{figure}

Dựa vào kết quả của phần mềm R ở hình \ref{ex2:model:12}, ta có $\hat{\beta_0} = -36.373$ và $\hat{\beta_2} = 17.464$, do đó ta có phương trình đường thẳng hồi quy theo mật độ của vật liệu ($X_2$) là:
\[Y = \hat{\beta_0} + \hat{\beta_2} X_2 = -36.373 + 17.464 X_2\]

\textbf{Mô hình 3:} $Y= \beta_0 + \beta_1 X_1 + \beta_2 X_2+ \epsilon$

Mô hình mặt phẳng hồi quy tương ứng: $Y= \hat{\beta_0} + \hat{\beta_1} X_1 + \hat{\beta_2} X_2$

\begin{figure}[h!]
	\centering
	\includegraphics[width=0.7\linewidth]{bai2_1iii.PNG} 
	\caption{Tham số mô hình 3}
	\label{ex2:model:13}
\end{figure}

Dựa vào kết quả của phần mềm R ở hình \ref{ex2:model:13}, ta có $\hat{\beta_0} = -30.081$, $\hat{\beta_1} = 4.905$ và $\hat{\beta_2} = 11.072$, do đó ta có phương trình mặt phẳng hồi quy theo độ dày của vật liệu ($X_1$) và mật độ của vật liệu ($X_2$) là:
\[Y = \hat{\beta_0} + \hat{\beta_1} X_1 + \hat{\beta_2} X_2 = -30.081 + 4.905 X_1 + 11.072 X_2\]

\subsubsection*{2. Xác định tỷ lệ phần trăm sự biến thiên của biến phụ thuộc cho từng mô hình có thể có trên.}

\textbf{Mô hình 1:} Dựa vào kết quả mô hình 1 (hình \ref{ex2:model:11}), hệ số xác định $R^2= 0.6903$ cho biết có 69.03\% sự thay đổi của mức độ bền dẻo của nhựa được giải thích bởi độ dày của vật liệu ($X_1$).

\textbf{Mô hình 2:} Dựa vào kết quả mô hình 2 (hình \ref{ex2:model:12}), hệ số xác định $R^2= 0.453$ cho biết có 45.3\% sự thay đổi của mức độ bền dẻo của nhựa được giải thích bởi mật độ của vật liệu ($X_2$).

\textbf{Mô hình 3:} Dựa vào kết quả mô hình 3 (hình \ref{ex2:model:13}), hệ số xác định $R^2= 0,8481$ cho biết có 84.81\% sự thay đổi của mức độ bền dẻo của nhựa được giải thích bởi hai yếu tố là độ dày vật liệu ($X_1$) và mật độ của vật liệu ($X_2$).

\subsubsection*{3. Nếu chúng ta chỉ quan tâm đến cả 2 biến giải thích, hãy lập bảng ANOVA?}

\begin{figure}[h!]
	\centering
	\includegraphics[width=0.7\linewidth]{bai2_3.PNG}
	\caption{Bảng ANOVA cho hai biến $X_1,X_2$}
	\label{ex2:model:anova3}
\end{figure}

Bảng ANOVA được biểu diễn trong hình \ref{ex2:model:anova3}.

\subsubsection*{4. Kiểm định giả thiết sau với mức ý nghĩa 5\%}
\[H_0 : \beta_1 = \beta_2= 0\]

\begin{figure}[h!]
	\centering
	\includegraphics[width=0.7\linewidth]{bai-2-4-summary}
	\label{fig:bai-2-4-summary}
\end{figure}

Từ hình trên, ta có giá trị thống kê Fisher $F_{obs} = 25.12 \sim F_{0,95}(2,9)$, tính được $p\_value = 0.0002075386$ qua hàm \textbf{pf}:
\begin{figure}[h!]
	\centering
	\includegraphics[width=0.7\linewidth]{bai-2-4-p_value}
	\label{fig:bai-2-4-pvalue}
\end{figure}

Vì $p\_value = 0.0002075386 < \alpha = 0.05$, suy ra bác bỏ giả thiết $H_0$.

\subsubsection*{5. Xác định khoảng tin cậy với mức ý nghĩa 5\% cho $\beta_1$ trong trường hợp mô hình chỉ có biến độc lập là độ dày của vật liệu ($X_1$).}

\begin{figure}[h!]
	\centering
	\includegraphics[width=0.7\linewidth]{bai-2-5-confint-beta1}
	\label{fig:bai-2-5-confint-beta1}
\end{figure}

Khoảng tin cậy $\beta_1$ với mức ý nghĩa 5\% của mô hình có một biến độc lập $X_1$ là $[3.187036,8.88479]$.

\subsubsection*{6. Với khoảng tin cậy vừa tìm được ở câu 5, chúng ta có thể khẳng địng rằng hồi quy tuyến tính là có ý nghĩa giữa mức độ bền dẻo của nhựa và độ dày của vật liệu và mật độ của vật liệu không? Chứng minh điều khẳng định của bạn.}

Nếu chỉ dựa vào khoảng tin cậy của $\beta_1$ ở câu 5, chúng ta \textbf{không} thể khẳng định được điều trên, vì kết quả ở câu 5 chỉ cho ta biết mối quan hệ tuyến tính giữa mức độ bền dẻo ($Y$) với độ dày của vật liệu ($X_1$).

Để chứng minh, ta lần lượt thực hiện kiểm định các giả thuyết sau:
\begin{enumerate}
	\item[i)] $H_0 : \beta_1 = \beta_2 = 0$ cho mô hình $Y = \beta_0 + \beta_1 X_1 + \beta_2 X_2 + \epsilon $
	
	Với kết quả từ câu 4, $H_0$ cũng bị bác bỏ do chứng minh trên, nghĩa là tồn tại tham số $\beta_1$ hoặc $\beta_2$ trong mô hình hồi quy hai biến $X_1, X_2$.
	
	\item[ii)] $H_0 : \beta_1 = 0$ cho mô hình $Y = \beta_0 + \beta_1 X_1 + \epsilon $
	
	Với kết quả từ câu 5, $H_0$ bị bác bỏ do $\beta_1 = 0$ không thuộc khoảng tin cậy $[3.187036,8.88479]$, nghĩa là tồn tại tham số $\beta_1$ trong mô hình hồi quy hai biến $X_1, X_2$.
	
	\item[iii)] $H_0 : \beta_2 = 0$ cho mô hình $Y = \beta_0 + \beta_1 X_1 + \beta_2 X_2 + \epsilon $
	
	Ta thực hiện kiểm định sự tồn tại tham số $\beta_2$ trong mô hình hồi quy hai biến $X_1, X_2$ khi đã có $X_1$.
	
	Đặt giả thuyết kiểm định:
	\[\begin{cases}
		H_0 : \beta_2 = 0 \text{ hay } Y = \beta_0 + \beta_1 X_1 + \epsilon \\
		H_1 : \beta_2 \ne 0 \text{ hay } Y = \beta_0 + \beta_1 X_1 + \beta_2 X_2 + \epsilon 
	\end{cases}\]
	
	Thực hiện kiểm định Fisher từng phần, ta có thống kê của kiểm định: 
	
	$$F_{obs} = \displaystyle \frac{\left [ SSE (H_0) - SSE(H_1) \right ] / r}{SSE(H_1)/(n-p)}  \sim F_{(r,n-p)} \text{ khi } H_0 \text{ đúng},$$
	trong đó $r = 1, n = 12, p = 3$.
	
	Từ bảng ANOVA ở câu 3, ta có được các giá trị sau:
	\begin{align*}
		&SST = 1420.667\\
		&SSR_{X_1} = 980.63\\
		&SSR_{X_2|X_1} = 224.22
	\end{align*}
	
	Trước tiên, ta cần tính $SSE (H_0)$ và $SSE(H_1)$:
	\begin{align*}
		&SSE(H_0) = SSE_{X_1} = SST - SSR_{X_1} = 440.032\\
		&SSE(H_1) = SSE_{X_2|X_1} = SST - SSR_{X_1} - SSR_{X_2|X_1} = 215.81
	\end{align*}
	
	Giá trị thống kê $$F_{obs} = 9.350885$$
	
	Với mức ý nghĩa $\alpha = 0.05$, tra bảng thống kê Fisher ta được:
	$$F_{1-\alpha}(r,n-p) = F_{0.95}(1,9) = 5.1174$$
	
	Vì $F_{obs} > F_{0.95}(1,9)$ nên ta bác bỏ $H_0$ với mức ý nghĩa 5\%, vậy tồn tại tham số $\beta_2$ trong mô hình hồi quy hai biến $X_1, X_2$.
\end{enumerate}

Từ kết luận của các giả thuyết trên, ta suy ra được mức độ bền dẻo ($Y$) có quan hệ tuyến tính chặt chẽ với cả hai biến độc lập là độ dày của vật liệu ($X_1$) và mật độ của vật liệu ($X_2$).

\newpage
\section*{BÀI 3}

\subsubsection*{1. Viết các mô hình tuyến tính với 2 biến độc lập (có thể).}

\begin{itemize}
	\item Mô hình với hai biến $x_1, x_2$
	\begin{equation}\label{model:12}
		y = \beta_0 + \beta_1x_1 + \beta_2x_2
	\end{equation}
	\item Mô hình với hai biến $x_1, x_3$
	\begin{equation}\label{model:13}
		y = \beta_0' + \beta_1'x_1 + \beta_3'x_3
	\end{equation}
	\item Mô hình với hai biến $x_2, x_3$
	\begin{equation}\label{model:23}
		y = \beta_0'' + \beta_2''x_2 + \beta_3''x_3
	\end{equation}
\end{itemize}

\subsubsection*{2. Ước lượng các hệ số hồi quy trong từng mô hình tuyến tính ở câu 1.}

\begin{itemize}
	\item Mô hình \ref{model:12}
	\begin{figure}[h!]
		\centering
		\includegraphics[width=0.7\linewidth]{bai-3-5-model-1}
		\caption{Tham số mô hình \ref{model:12}}
		\label{fig:bai-3-5-model-1}
	\end{figure}
	
	Hệ số hồi quy:
	\[\beta_0=25.84214, \beta_1=0.7148959, \beta_2=-0.3281129\]
	
	\item Mô hình \ref{model:13}
	\begin{figure}[h]
		\centering
		\includegraphics[width=0.7\linewidth]{bai-3-5-model-2}
		\caption{Tham số mô hình \ref{model:13}}
		\label{fig:bai-3-5-model-2}
	\end{figure}
	
	Hệ số hồi quy:
	\[\beta_0'=8.609241, \beta_1'=0.9272087, \beta_3'=0.02323681\]
	
	\item Mô hình \ref{model:23}
	\begin{figure}[h]
		\centering
		\includegraphics[width=0.7\linewidth]{bai-3-5-model-3}
		\caption{Tham số mô hình \ref{model:23}}
		\label{fig:bai-3-5-model-3}
	\end{figure}

	Hệ số hồi quy:
	\[\beta_0''=31.97642, \beta_2''=-0.4538954, \beta_3''=0.01996295\]
\end{itemize}

\subsubsection*{3. Với độ tin cậy 95\%, tìm khoảng tin cậy cho các tham số trong mô hình với 2 biến độc lập $x_1$ và $x_2$.}


\subsubsection*{4. Xác định hệ số xác định cho mỗi mô hình trong câu 1.}

\begin{itemize}
	\item Mô hình \ref{model:12} có bảng ANOVA:
	\begin{figure}[h]
		\centering
		\includegraphics[width=0.7\linewidth]{bai-3-5-anova-1}
		\caption{Bảng ANOVA của mô hình \ref{model:12}}
		\label{fig:bai-3-5-anova-1}
	\end{figure}
	
	Hệ số xác định:
	\[R^2 = \dfrac{SSR}{SST} = 0.6875395\]
	
	\item Mô hình \ref{model:13} có bảng ANOVA:
	\begin{figure}[h]
		\centering
		\includegraphics[width=0.7\linewidth]{bai-3-5-anova-2}
		\caption{Bảng ANOVA của mô hình \ref{model:13}}
		\label{fig:bai-3-5-anova-2}
	\end{figure}
	
	Hệ số xác định:
	\[R'^2 = \dfrac{SSR}{SST} = 0.5263211\]
	
	\item Mô hình \ref{model:23} có bảng ANOVA:
	\begin{figure}[h]
		\centering
		\includegraphics[width=0.7\linewidth]{bai-3-5-anova-3}
		\caption{Bảng ANOVA của mô hình \ref{model:23}}
		\label{fig:bai-3-5-anova-3}
	\end{figure}
	
	Hệ số xác định:
	\[R''^2 = \dfrac{SSR}{SST} = 0.4880253\]
\end{itemize}


\subsubsection*{5. Trong các mô hình trên, mô hình nào thích hợp nhất để giải thích sự biến thiên của $Y$ ?}

Với kết quả từ câu 4, ta có được thứ tự tăng dần các hệ số xác định từ các mô hình của $Y$ là \[R''^2 < R'^2 < R^2 \]

Vậy với hệ số $R^2$ cao nhất thì mô hình hai biến độc lập $x_1, x_2$ là phù hợp nhất để giải thích sự biến thiên của $Y$.


\subsubsection*{6. Viết mô hình tuyến tính dưới dạng ma trận với số biến độc lập nhiều nhất có thể, và xác định kích thước của ma trận.}

$$\mathbf{Y}_{14\times1} = \mathbf{X}_{14\times4}\beta_{4\times1} + \epsilon_{14\times1}$$

\[{Y_{14 \times 1}} = \left[ {\begin{array}{*{20}{c}}
  {{y_1}} \\ 
   \vdots  \\ 
  {{y_{14}}} 
\end{array}} \right];{X_{14 \times 4}} = \left[ {\begin{array}{*{20}{c}}
  1&{{x_{1,1}}}&{{x_{1,2}}}&{{x_{1,3}}} \\ 
  1& \vdots & \vdots & \vdots  \\ 
  1&{{x_{14,1}}}&{{x_{14,2}}}&{{x_{14,3}}} 
\end{array}} \right];{\beta _{4 \times 1}} = \left[ {\begin{array}{*{20}{c}}
  {{\beta _0}} \\ 
   \vdots  \\ 
  {{\beta _3}} 
\end{array}} \right];{\varepsilon _{14 \times 1}} = \left[ {\begin{array}{*{20}{c}}
  {{\varepsilon _1}} \\ 
   \vdots  \\ 
  {{\varepsilon _{14}}} 
\end{array}} \right]\]

\subsubsection*{7. Ước lượng các hệ số hồi quy trong mô hình tuyến tính ở câu 6.}
Viết lại mô hình: $Y = \beta_0 + \beta_1 X_1 + \beta_2 X_2 + \beta_3 X_3 + \epsilon$

\begin{center}
\includegraphics{bai3_6.PNG} 
\end{center}
Dựa vào kết quả mô hình, ta có các hệ số hồi quy:
$$\hat{\beta_0} = 32.89132, \hat{\beta_1} = 0.8019, \hat{\beta_2} = -0.38136, \hat{\beta_3} = -0.03713$$
\subsubsection*{8. Trong mô hình tuyến tính ở câu 6, tính ước lượng của $\mathbb{V}(\epsilon)$ và $\mathbb{V}(\hat{\beta})$.}


\subsubsection*{9. Với độ tin cậy 95\%, tìm khoảng tin cậy cho $\mathbb{V}(\epsilon)$. }


\subsubsection*{10. Khi thêm 2 biến độc lập $x_3$ và $x_2$ vào mô hình chỉ với 1 biến độc lập $x_1$ thì làm cho chất
lượng ước lượng cao hơn không?}
\end{document}