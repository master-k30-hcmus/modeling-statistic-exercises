\documentclass[a4paper]{book}
\usepackage[utf8]{vietnam}
\usepackage{scrextend}
\changefontsizes{13pt}
\usepackage{xcolor}
\usepackage{titlesec}
\usepackage{mdframed}
\usepackage{amsmath}
\usepackage{placeins}
\usepackage{array}
\usepackage[amsmath,standard,thmmarks]{ntheorem}
\usepackage{amssymb}
\usepackage{exscale}
\usepackage{amsfonts}
\usepackage{eucal}
\usepackage{enumerate}
\usepackage{enumitem}
\usepackage{commath}
\usepackage{graphicx}
\usepackage{tcolorbox}
\usepackage{subfig}
\usepackage{url}
\usepackage[unicode]{hyperref}
\newmdenv[linecolor=black,skipabove=\topsep,skipbelow=\topsep,
leftmargin=-5pt,rightmargin=-5pt,
innerleftmargin=5pt,innerrightmargin=5pt]{mybox}
\setlength{\parindent}{0pt}
\usepackage[left=2cm,right=2cm,top=2.5cm,bottom=2.5cm]{geometry}
\renewcommand{\baselinestretch}{1.5}
\newcommand{\heva}[1]{\left\{ 
	\begin{aligned}#1\end{aligned}\right.}
\usepackage{fancyhdr}
\pagestyle{fancy}
\fancyhf{}
\lhead{Mô hình hóa Thống kê}
\cfoot{\thepage}
\rhead{Nhóm 4}
\title{Tiểu luận cuối kỳ}

\allowdisplaybreaks

\author{NHÓM 4}

\date{\today}%

\begin{document}
	
	\begin{titlepage}
		\thispagestyle{empty}
		\begin{center}
			\textbf{\large{ĐẠI HỌC QUỐC GIA TP. HỒ CHÍ MINH\\TRƯỜNG ĐẠI HỌC KHOA HỌC TỰ NHIÊN}}\\
			---------------*---------------\\
			\vspace*{5.5cm}
			{\textcolor[rgb]{0.0,0.0,1.0}{\textbf{\Large{TIỂU LUẬN CUỐI KÌ}}}}\\
			\vspace{1cm}
			\textbf{\huge{\textcolor[rgb]{1.0,0.0,0.0}{MÔ HÌNH HÓA THỐNG KÊ}}}\\
			\vspace*{4cm}
			$\begin{array}{rll}
				\text{Giảng viên hướng dẫn:} &\text{\bf TS. Nguyễn Thị Mộng Ngọc} &  \\
				\text{Nhóm thực hiện:}     & \text{\textbf{Nhóm 4}} & \\
				\text{Học viên:} & \text{\textbf{Phan Thị Thùy An}} &\text{MSHV: 20C29002} \\
				& \text{\textbf{Đinh Thị Nữ }} &\text{MSHV: 20C29013} \\
				& \text{\textbf{Lý Phi Long}} &\text{MSHV: 20C29028} \\
				& \text{\textbf{Đặng Khánh Thi}} &\text{MSHV: 20C29038} 
			\end{array}$\\
			\vfill
			\normalsize{TP. Hồ Chí Minh $-$ Tháng 04, 2021}
		\end{center}
	\end{titlepage}
\tableofcontents
\section{Dữ liệu 1: Mô hình hồi quy đa biến}

\subsection*{Giới thiệu bộ dữ liệu}
Bộ dữ liệu được tìm thấy trên trang Kaggle - một cộng đồng trực tuyến về khoa học dữ liệu và học máy. Đó là bộ dữ liệu \textbf{Chi phí Y tế Cá nhân} \footnote{\url{https://www.kaggle.com/mirichoi0218/insurance}} (\textit{Medical Cost Personal Datasets}). Đây là một bộ dữ liệu được lấy ra từ cuốn \textit{Machine Learning with R} của Brett Lantz, một cuốn sách giới thiệu về học máy bằng R.

Bộ dữ liệu ghi lại các thông tin về thông tin của người đăng kí bảo hiểm và chi phí mà bảo hiểm y tế phải chi trả cho cá nhân người đó. Bộ dữ liệu có 1338 quan trắc, gồm 7 biến sau:
\begin{enumerate}
	\item \textbf{Age}: age of primary beneficiary
	\item \textbf{Sex}: giới tính của người đăng kí bảo hiểm
	\item \textbf{BMI}: Body mass index, providing an understanding of body, weights that are relatively high or low relative to height,	objective index of body weight ($kg / m ^ 2$) using the ratio of height to weight, ideally 18.5 to 24.9
	\item \textbf{Children}: số lượng trẻ em phụ thuộc được bao gồm bởi bảo hiểm y tế.	
	\item \textbf{Smoker}: 1 nếu người đó có hút thuốc, ngược lại là 0.
	\item \textbf{Region}: the beneficiary's residential area in the US, northeast, southeast, southwest, northwest.
	\item \textbf{Charges}: Chi phí y tế chi trả cho cá nhân được chi bởi bảo hiểm y tế.
\end{enumerate}

\subsection*{Phân tích và chọn mô hình}

\subsection*{Nhận xét và kết luận}
\section{Dữ liệu 2: Hồi quy thành phần chính}

\subsection*{Giới thiệu bộ dữ liệu}
Hiện nay, Xe đạp cho thuê được giới thiệu ở nhiều thành phố để nâng cao sự thoải~mái khi di chuyển. Điều cần quan tâm khi cho thuê xe đạp là xe đạp phải luôn sẵn sàng và tiếp cận được người dùng vào đúng thời điểm, giúp giảm bớt thời gian chờ. Do đó, việc đảm bảo một nguồn cung cấp xe đạp cho thuê ổn định cho thành phố trở thành mối~quan~tâm lớn. Phần quan trọng là cần dự đoán được số lượng xe đạp cần thiết tại mỗi~giờ, để có được nguồn cung cấp xe đạp cho thuê ổn định.

\textbf{Bộ dữ liệu:} Nhu cầu thuê xe đạp ở Seoul\footnote{\url{https://archive.ics.uci.edu/ml/datasets/Seoul+Bike+Sharing+Demand}} (\textbf{Seoul Bike Sharing Demand Dataset}) ghi lại các thông tin về thời tiết, số lượng xe đạp được thuê mỗi giờ theo từng ngày, từ 01/12/2017 đến 31/11/2018. Bộ dữ liệu có 8760 quan trắc, gồm 14 biến:
\begin{enumerate}
	\item \texttt{Date} - Ngày ghi lại số lượng xe đạp cho thuê
	\item \texttt{Rented Bike count} - Số lượng xe đạp được thuê được ghi lại theo mỗi giờ
	\item \texttt{Hour} - Giờ trong ngày
	\item \texttt{Temperature} - Nhiệt độ ($^o C$)
	\item \texttt{Humidity} - Độ ẩm (\%)
	\item \texttt{Windspeed} - Tốc độ gió ($m/s$)
	\item \texttt{Visibility} - Tầm nhìn xa ($10m$)
	\item \texttt{Dew point temperature} - Nhiệt độ điểm sương ($^o C$)
	\item \texttt{Solar radiation} - Bức xạ mặt trời ($Mj/m^2$)
	\item \texttt{Rainfall} - Lượng mưa ($mm$)
	\item \texttt{Snowfall} - Lượng tuyết rơi ($cm$)
	\item \texttt{Seasons} - Mùa (Winter, Spring, Summer, Autumn)
	\item \texttt{Holiday} - Ngày lễ (Holiday/No holiday)
	\item \texttt{Functional Day} - Ngày làm việc (Yes nếu là ngày làm việc, No nếu ngược lại)
\end{enumerate}

Một vài quan trắc đầu tiên trong bộ dữ liệu được thể hiện trong hình \ref{A2_head}

\begin{figure}[H]
	\centering
	\includegraphics[width=0.7\linewidth]{../Photo Of Result/A2_head}
	\caption{Một vài quan trắc đầu tiên và số chiều của bộ dữ liệu}
	\label{A2_head}
\end{figure}

Vì mục đích bài toán là dự đoán số lượng xe đạp theo mỗi giờ, do đó nhóm em loại bỏ biến \texttt{Date}. Bên cạnh đó, các biến định tính cũng được biến đổi thành các biến dummy, cụ thể: \texttt{Hour} được phân rã thành 24 biến, \texttt{Seasons} được phân rã thành 4 biến, \texttt{Holiday} mang giá trị 1 nếu là Holiday và 0 nếu ngược lại, \texttt{Functional Day} mang giá~trị 1 nếu là Yes và 0 nếu ngược lại. Lúc này bộ dữ liệu gồm 39 biến.

\begin{figure}[H]
	\centering
	\includegraphics[width=0.7\linewidth]{../Photo Of Result/A2_dummy}
	\caption{Dữ liệu sau khi loại bỏ \texttt{Date} và tạo các biến giả}
	\label{A2_head2}
\end{figure}

\subsection*{Phân tích và chọn mô hình}

\begin{figure}[H]
	\centering
	{\includegraphics[width=.8\linewidth]{../Photo Of Result/A2_plotvar1}}\\
	{\includegraphics[width=.8\linewidth]{../Photo Of Result/A2_plotvar2}}
	\caption{Quan sát phân bố của từng biến với \texttt{Count}}
	\label{A2_visual1}
\end{figure}

Các kết quả từ hình \ref{A2_visual1} cho thấy xe đạp được thuê nhiều khi nhiệt độ (\texttt{Temp}) và nhiệt độ điểm sương (\texttt{Dew}) 

\begin{figure}[H]
	\centering
	{\includegraphics[width=\linewidth]{../Photo Of Result/A2_corr}}	
	\caption{Quan sát ma trận hiệp phương sai và các biến định tính}
	\label{A2_visual2}
\end{figure}

%Từ những quan sát về kiểu dữ liệu và mối tương quan giữa các biến độc lập với nhau cũng như giữa các biến độc lập và biến phụ thuộc ta thấy: Mô hình nhóm đi phân tích gồm 8 biến độc lập liên tục, 4 biến độc lập kiểu phân loại. Một trong số chúng có mối tương quan mạnh mẽ với nhau như biến \texttt{Dew point temperature} và hai biến \texttt{Temperature}, \texttt{Humidity}. mặt khác,  Để giải quyết vấn đề của bộ dữ liệu có số biến lớn và có mối tương quan mạnh giữa các biến độc lập với nhau nhóm sẽ dùng phương pháp tiếp cận giảm chiều dữ liệu (PCA) để biến đổi dữ liệu về không gian có số chiều nhỏ hơn mà vẫn giữ được nhiều thông tin nhất có thể của bộ dữ liệu.

\begin{figure}[H]
	\centering
	\includegraphics[width=0.9\linewidth]{../Photo Of Result/A2_PCproportion.PNG}
	\caption{Kết quả PCA}
	\label{A2_PCAmod1}
\end{figure}

\begin{figure}[H]
	\centering
	\includegraphics[width=0.9\linewidth]{../Photo Of Result/A2_pca_plotvar.png}
	\caption{Thành phần chính}
	\label{A2_Var}
\end{figure}


%Dựa vào kết quả PCA được xuất ra từ phần mềm R ta thấy 26 thành phần chính đầu tiên giải thích được khoảng 85 $\%$ tập dữ liệu ban đầu. Thực hiện hồi quy tuyến tính trên những thành phần chính này thu được kết quả hồi quy như hình \ref{A2_model26PCA}. \\


%Mô hình giải thích được khoảng 50 $\%$ ($R^{2}$) cho sự thay đổi số lượng thuê xe đạp tại Seoul. Tuy nhiên trong mô hình hồi quy vẫn chưa những biến \texttt{PC7}, \texttt{PC13}, và \texttt{PC14} không có ý nghĩa thống kê do $\rho_{value} \ge \alpha$ hình \ref{A2_model26PCA}.

%Tiến hành loại bỏ những biến này ra và thực hiện hồi quy tuyến tính trên tập biến mới hình \ref{A2_summaryPCA2}. $R^{2} \approx 49.5 \% $ không thay đổi nhiều so với mô hình trước đó. Tuy nhiên trong mô hình mới này tất cả những biến độc lập đều có ý nghĩa thống kê do $\rho_{value} \ge \alpha$.

\begin{figure}[H]
	\centering
	\subfloat[Hồi quy với 26 thành phần chính đầu tiên]
	{\includegraphics[width=0.5\linewidth]{../Photo Of Result/A2_pca_mod1.PNG}} \hfill
	\subfloat[Hồi quy với 23 thành phần chính]
	{\includegraphics[width=0.5\linewidth]{../Photo Of Result/A2_pca_mod2.PNG}}
	\caption{Hồi quy thành phần chính}
	\label{A2_modPCA}
\end{figure}


\subsection*{Nhận xét và kết luận}
\section{Dữ liệu 1}
Những thông tin vê các giám đốc điều hành các tập đoàn Hoa Kỳ. Bộ dữ liệu gồm 177 quan trắc và 15 biến.

\subsection*{Tìm hiểu và tiền xử lý dữ liệu}

Một số biến trong bộ dữ liệu kiểu số có đơn vị tính lớn như: $sales'$, $profits$, $lmktval$. Nếu đưa những biến này vào phương trình hồi quy có thể dẫn tới hiện tượng bias do tác~động của những biến này lên model lấn át những biến khác còn lại như $age$, $ceoten$.... Nên ta sẽ dùng phương pháp logarit cho 3 biến này trong model tương ứng với 3 biến mới là:  $lsales''$, $lmktval$   và $profmarg$. (1)


Từ biểu đồ dưới ta thấy ba biến định lượng $\textit{lsales}$, $\textit{lmktval}$ và $\textit{profmarg}$ xảy ra hiện~tượng đa cộng tuyến.
Tuy nhiên có xảy ra hiện tượng đa cộng tuyến giữa 2 biến sales và profit luôn (hình \ref{fig-b1:plot-vars}).

Tính độ correlation của biến $salary$ với lần lượt 2 biến trên ta có:

\begin{figure}[H]
	\centering
	\includegraphics[width=.7\linewidth]{../Photo Of Result/B1_plotVriables.png}  
	\caption{Mối tương quan giữa các biến}
	\label{fig-b1:plot-vars}
\end{figure}

\begin{figure}[H]
	\centering
	\includegraphics[scale = 0.6]{../Photo Of Result/B1_CorTable.PNG}  
	\caption{Mức độ tương quan giữa biến lsales và promarg Correlation}
	\label{fig-b1:corr-table}
\end{figure}

Xét bảng correlation giữa các biến độc lập với nhau và giữa các biến độc lập với biến phụ thuộc, ta thấy: Giữa hai biến $\textit{lmktval}$ và biến $\textit{lsales}$ có mối tương quan rất cao ($\approx$ 0.75). Tuy nhiên biến $\textit{lmktval}$ lại có mối tương quan cao hơn với biến phụ thuộc $\textit{salary}$. Mặt khác giữa biến $\textit{profmarg}$ và $\textit{lsales}$ cũng có mối tương quan cao ($\approx$ -0.42). Nên ta loại bỏ biến $\textit{lsales}$ khỏi danh sách các biến được xét. (2)

Từ (1) và (2) ta có mô hình với đầy đủ các biến cần lựa chọn như sau:
\begin{equation}\label{eq-b1:full-model}
	\begin{split}
		salary 	= \beta_0 + &\beta_1*age + \beta_2*college + \beta_3*grad + \beta_4*comten\\
		&+ \beta_5*ceoten + \beta_6*lmktval + \beta_7*profmarg
	\end{split}
\end{equation}


Thực hiện phân rã hai biến phân loại gồm $college$ và $grad$ trước khi thực hiện phương~pháp chọn biến \textbf{Stepwise} \textbf{tiến} với \textbf{tiêu chuẩn AIC}.

Để đánh giá chất lượng mô hình ta chia tâp dữ liệu thành hai phần, training và testing, với tỷ lệ $80:20$ sau đó tiến hành phương pháp chọn biến trên tập training.

\subsection*{Thực hiện chọn biến bằng phương pháp StepWise tiến và tiêu chuẩn AIC}

%\begin{figure}[H]
%	\centering
%	\includegraphics[scale = 0.52]{../Photo Of Result/B1_stepwiseForward.PNG}  
%	\caption{Kết quả chọn biến theo phương pháp StepWise tiến với tiêu chuẩn AIC}
%	\label{fig-b1:stepwise-forward}
%\end{figure}

Tổng quan tiêu chuẩn AIC thì mô hình tốt là mô hình có giá trị AIC nhỏ nhất. Ở mô hình 1, biến $\textit{lmktval}$ được chọn vào mô hình vì có AIC nhỏ nhất trong tất cả các kết~hợp với các biến còn lại. Tương tự AIC được tính cho mô hình thêm biến thứ 2, $\textit{ceoten}$, và biến thứ 3 là $\textit{ceoten}$ (hình \ref{ex1:model:1}).

\begin{figure}[H]
	\centering
	\includegraphics[width=.7\linewidth]{../Photo Of Result/B1_summary.PNG}  
	\caption{Kết quả hồi quy mô hình với các biến được chọn}
	\label{ex1:model:1}
\end{figure}

Với ba biến được chọn ở trên, mô hình \ref{eq-b1:full-model} trở thành mô hình mới:
\begin{equation}\label{1.2}
\textit{salary} = -950.6 + 248.2 * \textit{lmktval} - 13.9 *\textit{profmarg} + 11.7  *\textit{ceoten}
\end{equation}
Tuy nhiên ta nhận thấy biến $\textit{ceoten}$ có $\rho_{value} \ge \alpha$ (0.05738 $\ge$ 0.05) nên không có ý~nghĩa thống kê trong mô hình. Ta tiến hành bỏ biến $\textit{ceoten}$ và hồi quy mô hình với hai biến còn lại kết quả thu được từ phần mềm R như hình \ref{fig-b1:new-summary}:

\begin{figure}[H]
	\centering
	\includegraphics[width=.7\linewidth]{../Photo Of Result/B1_newsummary.PNG}  
	\caption{Kết quả hồi quy mô hình với hai biến còn lại}
	\label{fig-b1:new-summary}
\end{figure}

Mô hình thống kê mới:
\begin{equation}\label{1.3}
\textit{salary} = -830.7 + 245.3 *\textit{lmktval} -13.9 *\textit{profmarg}
\end{equation}

Trường hợp này hai biến còn lại có ý nghĩa thống kê. Tuy nhiên mô hình được tạo bởi hai biến này chỉ giải thích được 23$\%$ sự biến thiên của biến phụ thuộc (hình \ref{fig-b1:new-summary}). Nguyên nhân dẫn tới kết quả thấp là do số lượng data ít, các biến giải thích ít không tạo nên mô hình đặc trưng được.

\subsection*{Test trên tập test và nhận xét kết quả}

Thực hiện dự đoán trên tập dữ liệu test từ kết quả mô hình \ref{1.3} và dùng chỉ số đánh giá MSE (trung bình bình phương sai số) ta có:

\begin{figure}[H]
	\centering
	\includegraphics[width=.5\linewidth]{../Photo Of Result/B1_MSE.PNG}  
	\caption{Chỉ số đo lường kết quả MSE}
	\label{fig-b1:mse}

Kết quả MSE $\approx$ 454097 lớn hơn nhiều so với giá trị Mean : 887.5 nên ta có thể thấy hai yếu tố gồm: giá thị trường ( \textit{lmktval} ) và tỷ lệ phần trăm lợi nhuận (\textit{profmarg}) là chưa đủ để giải thích mức độ tăng giảm của tiền lương của các giám đốc điều hành các tập đoàn Hoa Kỳ. Để cải thiện kết quả mô hình ta nên tiến hành thu thập thêm dữ liệu và tiến hành lựa chọn biến dựa trên dữ liệu mới này. Bên cạnh đó có thể xem xét tới xem xét tới các nhân tố khác ảnh hưởng tới tiền lương của các giám đốc Hoa kỳ như: Lĩnh vực hoạt động ( ngân hàng, hàng không, công nghệ, vận tải...); mức lương trước đó; số năm kinh nghiệm, giới tính,...
	

\end{figure}

\section{Dữ liệu 2}
Bộ dữ liệu ghi lại lịch sử về những ngôi nhà được bán từ 5/2014 đến 5/2015 ở quận King, bang Washington, Hoa Kỳ. Bộ dữ liệu bao gồm 21613 quan trắc, gồm 21 biến.
\subsection*{Tìm hiểu dữ liệu}
%một vài quan trắc đầu tiên, bảng correlation, quan sát phân bố biến Y
\begin{figure}[h!]
\centering
\subfloat[Một số quan trắc đầu tiên]
{\includegraphics[width=0.7\linewidth]{../Photo Of Result/B2_headdata}}\\
\subfloat[Hệ số tương quan giữa các biến]	{\includegraphics[width=.5\linewidth]{../Photo Of Result/B2_corr}}\hfill
\subfloat[Phân bố của biến phụ thuộc]
{\includegraphics[width=.5\linewidth]{../Photo Of Result/B2_histPrice}}\hfill
\caption{Một số quan sát ban đầu của bộ dữ liệu}
\end{figure}

Bộ dữ liệu cung cấp gồm 21 biến, trong đó biến \textbf{id} và \textbf{date} sẽ được loại bỏ khỏi dữ~liệu trước khi tiến hành phân tích, vì nhóm em nghĩ các biến này chỉ để ghi lại chỉ số và thời gian mua bán, không có ý nghĩa thống kê. 

\subsection*{Phân tích, chọn mô hình}
%mô hình ban đầu (Đầy đủ biến, chưa chuẩn hóa), mô hình sau khi chọn biến bằng các tiêu chuẩn ..., mô hình cuối cùng (nếu có chuẩn hóa dữ liệu)

%đưa ra kết quả R của mỗi mô hình, giải thích vì sao chọn mô hình cuối cùng, phân tích các kết quả từ plot xem mô hình có đảm bảo các giả thiết hay không (kỳ vọng = 0, phân phối chuẩn, phương sai không đổi, có mối quan hệ tuyến tính, không có đa cộng tuyến)
\begin{figure}[h!]
	\centering
	\includegraphics[width=0.7\linewidth]{../Photo Of Result/B2_originalmodel_R}
	\caption{Mô hình hồi quy đầy đủ ban đầu}
	\label{B2_full}
\end{figure}
$*$ \textbf{Phương pháp chọn: Stepwise - lùi; tiêu chuẩn chọn: BIC}.

Bộ dữ liệu (sau khi loại bỏ id và date) có 18 biến giải thích, do đó nhóm em chọn phương pháp lùi (\textbf{stepwise - backward}) cho bộ dữ liệu này. Trong mô hình hồi quy đầy đủ (Hình \ref{B2_full}), đa số các biến giải thích đều có ý nghĩa thống kê, do đó tiến hành phương~pháp lùi (loại biến dần dần) sẽ tiết kiệm thời gian hơn so với các phương pháp còn lại.

\begin{figure}[h!]
	\centering
	\includegraphics[width=0.7\linewidth]{../Photo Of Result/B2_BIC}
	\caption{Mô hình khi chọn bằng tiêu chuẩn BIC}
	\label{B2_BIC}
\end{figure}



\subsection*{Kết luận}
%Nhận xét các biến ảnh hưởng đến biến Y từ mô hình cuối cùng, ý nghĩa của mô hình đã chọn
\section{Dữ liệu 3}

\section{Dữ liệu 4}
Bộ dữ liệu ghi lại những yếu tố có thể ảnh hưởng đến lương (\$ giờ) của người lao động ở Anh năm 1976

\subsection*{Tìm hiểu dữ liệu}
Bộ dữ liệu gồm 13 biến sau:
\begin{itemize}
	\item \texttt{wage}: Số lượng trung bình một giờ
	\item \texttt{educ}: Số năm đào tạo
	\item \texttt{exper} Số năm kinh nghiệm tiềm năng
	\item \texttt{tenure} Số năm làm việc hiện tại
	\item \texttt{nonwhite} =1 nếu là người da màu
	\item \texttt{female} =1 nếu là phụ nữ
	\item \texttt{married} =1 nếu đã kết hôn
	\item \texttt{numdep} số người phụ thuộc
	\item \texttt{smsa} =1 nếu sống ở vùng đô thị Hoa Kỳ
	\item \texttt{northcen} =1 nếu sống ở phía Bắc trung tâm Hoa Kỳ
	\item \texttt{south} =1 nếu sống ở khu vực phía nam
	\item \texttt{west} =1 nếu sống ở khu vực phía tây
	\item \texttt{construc} =1 nếu làm việc ở construc. indus
	\item \texttt{ndurman} =1 nếu làm việc nondur. manuf. indus.
	\item \texttt{trcommpu} =1 nếu trong trans, commun, pub ut
	\item \texttt{trade} =1 nếu bán buôn hoặc bán lẻ
	\item \texttt{services} =1 nếu trong services indus
	\item \texttt{profserv} =1 nếu trong prof. serv. indus.
	\item \texttt{profocc} =1 nếu trong profess. occupation
	\item \texttt{clerocc} =1 nếu trong clerical occupation
	\item \texttt{servocc} =1 nếu trong  service occupation
	\item \texttt{lwage} log(wage)
	\item \texttt{expersq} exper2
	\item \texttt{tenursq} tenure2
\end{itemize}

\begin{figure}[H]
	\centering
	{\includegraphics[width=0.63\linewidth]{../Photo Of Result/describe(4)-1}}
\end{figure}
\begin{figure}[H]
	\centering
	{\includegraphics[width=0.63\linewidth]{../Photo Of Result/describe(4)-2}}
	{\includegraphics[width=0.63\linewidth]{../Photo Of Result/describe(4)-3}}
	\caption{Khái quát dữ liệu}
	\label{describe-4}
\end{figure}

Dựa vào kết quả mô tả dữ liệu từ R trong hình \ref{describe-4}, ta thấy được bộ dữ liệu gồm  24 biến và 526 quan trắc, các biến dữ liệu không bị missing value và được chia làm hai loại:
\begin{itemize}
	\item Các biến định tính: "nonwhite", "female", "married", "smsa"  ,   "northcen" , "south"  ,  "west"  ,   "construc" "ndurman"  "trcommpu" , "trade" ,   "services" , "profserv" , "profocc" , "clerocc" , "servocc" 
	\item Các biến định lượng: "wage"  ,  "educ"  ,  "exper" ,  "tenure" , "numdep" , "lwage" ,  "expersq", "tenursq"
\end{itemize}

\subsection*{Phân tích dữ liệu}
\begin{figure}[H]
	\centering
	\subfloat[Biểu đồ biến định lượng theo độ tương quan, sơ đồ cột, điểm dữ liệu]
	{\includegraphics[width=0.8\textwidth]{../Photo Of Result/Plot-dinhluong-data4}} \hfill
	\subfloat[Biểu đồ biến định tính theo độ tương quan, sơ đồ cột, điểm dữ liệu]
	{\includegraphics[width=0.85\textwidth]{../Photo Of Result/Plot-dinhtinh-data4}}
	\caption{Biểu đồ vẽ các biến theo độ tương quan, sơ đồ cột, điểm dữ liệu}
	\label{plot_data4}
\end{figure}

Dựa vào hình \ref{plot_data4}, ta thấy độ tương quan giữa ba cặp biến (\texttt{lwage}, \texttt{wage}), (\texttt{tenursq}, \texttt{tenure}, (\texttt{expersq}, \texttt{exper}) đều trên $0.9$ nên ta sẽ phải bỏ ba biến này ra khỏi bộ dữ liệu để tránh hiện tượng đa cộng tuyến. Nhìn vào bộ dữ liệu, ta nhận thấy được các biến 
\texttt{lwage}, \texttt{tenursq}, \texttt{expersq} đều là sự biến đổi từ ba biến \texttt{wage}, \texttt{tenure}, \texttt{exper} ban đầu nên ta sẽ quyết định bỏ hẳn 3 biến \texttt{lwage}, \texttt{tenursq}, \texttt{expersq} này ra khỏi bộ dữ liệu.

\subsection*{Chọn mô hình}
Ta sẽ xây dựng mô hình hồi quy tuyến tính đầy đủ

\begin{figure}[H]
	\centering
	\includegraphics[width=0.7\textwidth]{../Photo Of Result/full-model-data4}
	\caption{Kết quả hồi quy mô hình đầy đủ từ R}
	\label{full-model}
\end{figure}

Ta có thể thấy trong hình \ref{full-model} chỉ một vài biến như là \texttt{educ}, \texttt{tenure}, \texttt{female}, \texttt{trade}, \texttt{services}, \texttt{profocc} có ý nghĩa thống kê $0.001$. Do đó, ta cần sử dụng các phương pháp chọn biến để mô hình tốt hơn.
\subsubsection*{Hướng tiếp cận 1: Chọn tất cả}

\begin{longtable}{cllll}
	\hline
	\begin{tabular}[c]{@{}c@{}}Số lượng \\ biến\end{tabular} &
	\multicolumn{1}{c}{Biến dự đoán} &
	\multicolumn{1}{c}{$R^2_{adj}$} &
	\multicolumn{1}{c}{$C_p$} &
	\multicolumn{1}{c}{BIC} \\ \hline
	\endhead
	%
	\hline
	\endfoot
	%
	\endlastfoot
	%
	1 &
	profocc &
	\multicolumn{1}{c}{0.19362} &
	243.45475 &
	-101.6706 \\
	2 &
	educ, tenure &
	\multicolumn{1}{c}{0.29919} &
	143.97580 &
	-170.2154 \\
	3 &
	educ, tenure, female &
	\multicolumn{1}{c}{0.35396} &
	92.91189 &
	-207.7622 \\
	4 &
	educ, tenure, female, profocc &
	0.39249 &
	57.37900 &
	-234.8483 \\
	5 &
	educ, tenure, femanle, profocc, trade &
	0.41818 &
	34.07821 &
	-252.3208 \\
	6 &
	educ, tenure, female, profocc, trade, west &
	0.42657 &
	27.14121 &
	-254.7032 \\
	7 &
	\begin{tabular}[c]{@{}l@{}}educ, tenure, female, profocc, trade, west, \\ services\end{tabular} &
	0.43425 &
	20.89218 &
	-256.5481 \\
	8 &
	\begin{tabular}[c]{@{}l@{}}educ, tenure, female, profocc, trade, west,\\ services, smsa\end{tabular} &
	0.44033 &
	16.16804 &
	\textbf{-256.9875} \\
	9 &
	\begin{tabular}[c]{@{}l@{}}educ, tenure, female, profocc, trade, west,\\ services, smsa, married\end{tabular} &
	0.44469 &
	13.08489 &
	-255.8480 \\
	10 &
	\begin{tabular}[c]{@{}l@{}}educ, tenure, female, profocc, trade, west,\\ services, smsa, married, northcen\end{tabular} &
	0.44560 &
	13.22747 &
	-251.4683 \\
	11 &
	\begin{tabular}[c]{@{}l@{}}educ, tenure, female, profocc, trade, west,\\ services, smsa, married, ndurman, profserv\end{tabular} &
	0.44637 &
	13.50318 &
	-246.9594 \\
	12 &
	\begin{tabular}[c]{@{}l@{}}educ, tenure, female, profocc, trade, west,\\ services, smsa, married, ndurman, profserv,\\ trcommpu\end{tabular} &
	0.44827 &
	12.73335 &
	-243.5280 \\
	13 &
	\begin{tabular}[c]{@{}l@{}}educ, tenure, female, profocc, trade, west, \\ services, smsa, married, ndurman, profserv, \\ trcommpu, northcen\end{tabular} &
	0.44939 &
	12.69999 &
	-239.3528 \\
	14 &
	\begin{tabular}[c]{@{}l@{}}educ, tenure, female, profocc, trade, west,\\ services, smsa, married, ndurman, profserv,\\ trcommpu, northcen, numdep\end{tabular} &
	0.44959 &
	13.51543 &
	-234.3090 \\
	15 &
	\begin{tabular}[c]{@{}l@{}}educ, tenure, female, profocc, trade, west,\\ services, smsa, married, ndurman, profserv, \\ trcommpu, northcen, numdep, exper\end{tabular} &
	\textbf{0.45032} &
	13.84069 &
	-229.7754 \\
	16 &
	\begin{tabular}[c]{@{}l@{}}educ, tenure, female, profocc, trade, west,\\ services, smsa, married, ndurman, profserv,\\ trcommpu, nothcen, numdep, exper, south\end{tabular} &
	0.45015 &
	15.00327 &
	-224.3782 \\
	17 &
	\begin{tabular}[c]{@{}l@{}}educ, tenure, female, profocc, trade, west,\\ services, smsa, married, ndurman, profserv, \\ trcommpu, northcen, numdep, exper, construc,\\ clerocc\end{tabular} &
	0.45003 &
	16.12009 &
	-219.0300 \\
	18 &
	\begin{tabular}[c]{@{}l@{}}educ, tenire, female, profocc, trade, west,\\ services, smsa, married, ndurman, profserv, \\ trcommpu, northcen, numdep, exper, construc, \\ clerocc, south\end{tabular} &
	0.44988 &
	17.26616 &
	-213.6529 \\
	19 &
	\begin{tabular}[c]{@{}l@{}}educ, tenire, female, profocc, trade, west,\\ services, smsa, married, ndurman, profserv, \\ trcommpu, northcen, numdep, exper, construc,\\ clerocc, south, servocc\end{tabular} &
	0.44906 &
	\textbf{19.01455} &
	-207.6496 \\
	20 &
	\begin{tabular}[c]{@{}l@{}}educ, tenire, female, profocc, trade, west,\\ services, smsa, married, ndurman, profserv, \\ trcommpu, northcen, numdep, exper, construc,\\ clerocc, south, servocc, nonwhite\end{tabular} &
	0.44799 &
	21.00000 &
	-201.3995 \\ \hline
	\caption{Giá trị $R^2_{adj}$, $C_p$, BIC, cho từng tập biến con tốt nhất}
	\label{table-all-subset}\\
\end{longtable}

Dựa vào bảng \ref{table-all-subset} ta thấy được mô hình có chỉ số BIC tốt nhất là mô hình 8 biến. Tuy nhiên mô hình có hệ số xác định hiệu chỉnh tốt nhất là mô hình có 15 biến và mô hình có hệ số $C_p$ tốt nhất là mô hình 19 biến dự đoán


\subsubsection*{Hướng tiếp cận 2:: Phương pháp tiến dựa trên AIC}
\begin{figure}[H]
	\includegraphics[width=0.45\textwidth]{../Photo Of Result/stepAIC(4)-1}
	\includegraphics[width=0.45\textwidth]{../Photo Of Result/stepAIC(4)-2}
	\includegraphics[width=0.45\textwidth]{../Photo Of Result/stepAIC(4)-3}
	\includegraphics[width=0.45\textwidth]{../Photo Of Result/stepAIC(4)-4}
	\includegraphics[width=0.45\textwidth]{../Photo Of Result/stepAIC(4)-5}
	\includegraphics[width=0.45\textwidth]{../Photo Of Result/stepAIC(4)-6}
	\includegraphics[width=0.45\textwidth]{../Photo Of Result/stepAIC(4)-7}
	\includegraphics[width=0.45\textwidth]{../Photo Of Result/stepAIC(4)-8}
	\includegraphics[width=\textwidth]{../Photo Of Result/stepAIC(4)-9}
	\caption{Các biến định tính được vẽ ra}
	\label{stepAIC}
\end{figure}

Dự vào hình \ref{stepAIC}, sau khi chạy code R phương pháp lùi dựa theo tiêu chí AIC thì mô hình chọn được là mô hình gồm 11 biến

\begin{equation*}
	\begin{multlined}
		\texttt{wage } = \beta_0 + \text{ } \beta_{educ}\times \texttt{educ} \text{ } + \text{ } \beta_{tenure} \times \texttt{tenure} \text{ }+\text{ }\beta_{female} \times \texttt{female} \text{ } \\
		+ \beta_{married} \text{ } \times \texttt{married} \text{ } +\text{ }\beta_{smsa} \times\texttt{smsa}\text{ } +\text{ }\beta_{northcen}\times \texttt{northcen} \\
		+\text{ }\beta_{west} \times \texttt{west} \text{ } + \text{ }\beta_{ndurman}\times \texttt{ndurman}\text{ } +\text{ }\beta_{trcommpu} \times\texttt{trcommpu}\text{ } \\
		+\text{ }\beta_{trade}\times \texttt{trade} +\text{ }\beta_{services} \times \texttt{services} \text{ } 
		+ \text{ }\beta_{profserv} \times \texttt{profserv} \text{ } + \text{ }\beta_{profocc}\times \texttt{profocc}
	\end{multlined}
\end{equation*}

\subsubsection*{Hướng tiếp cận 3: Phương pháp Stagewise}
\begin{figure}[H]
	\centering
	\subfloat[Biểu đồ chọn mô hình hồi quy]
	{\includegraphics[width=.7\linewidth]{../Photo Of Result/stagewise plot}} \hfill
	\subfloat[Bảng hệ số của mô hình hồi quy]
	{\includegraphics[width=.8\linewidth]{../Photo Of Result/coef-stagewise-4}}
	\caption{Biểu đồ Stagewise cho mô hình hồi quy}
	\label{stagewise}
\end{figure}

Dựa vào hình \ref{stagewise}, sau khi chạy code R, phương pháp Stagewise đưa ra đề xuất model gồm 11 biến:

\begin{equation*}
	\begin{multlined}
		\texttt{wage } = \beta_0 + \text{ } \beta_{educ}\times \texttt{educ} \text{ } + \text{ } \beta_{tenure} \times \texttt{tenure} \text{ }+\text{ }\beta_{female} \times \texttt{female} \text{ } 
		+ \beta_{married} \text{ } \times \texttt{married} \text{ } \\
		+\text{ }\beta_{smsa} \times\texttt{smsa}\text{ }  
		+ \text{ }\beta_{west} \times \texttt{west} \text{ } 
		+ \text{ }\beta_{trade}\times \texttt{trade} + \text{ }\beta_{services} \times \texttt{services} \text{ } \\
		+ \text{ }\beta_{servocc} \times \texttt{servocc} \text{ } + \text{ }\beta_{profocc}\times \texttt{profocc}
	\end{multlined}
\end{equation*}

Dựa vào cả ba hướng tiếp cận, ta sẽ xây dựng 5 mô hình hồi quy lại theo các phương pháp chọn biến

\begin{figure}[H]
	\begin{tabular}{cc}
		\centering
		\includegraphics[width=0.4\textwidth]{../Photo Of Result/model-R-4} &   \includegraphics[width=0.45\textwidth]{../Photo Of Result/model-BIC-4} \\
		(a) Mô hình hồi quy theo giá trị $R^2_{adj}$  & (b) Mô hình hồi quy theo giá trị BIC \\[6pt]
		\includegraphics[width=0.45\textwidth]{../Photo Of Result/model-Cp-4} &   \includegraphics[width=0.45\textwidth]{../Photo Of Result/model-stepwise-aic-4} \\
		(c) Mô hình hồi quy theo giá trị $C_p$ & (d) Mô hình hồi quy Stepwise dựa trên AIC \\[6pt]
		\multicolumn{2}{c}{\includegraphics[width=0.45\textwidth]{../Photo Of Result/model-stagewise-4} }\\
		\multicolumn{2}{c}{(e) Mô hình hồi quy theo phương pháp Stagewise}
	\end{tabular}
	\caption{Mô hình hồi quy dựa trên các phương pháp chọn biến}
	\label{5-model-reg}
\end{figure}

Dựa vào kết quả R trong hình \ref{5-model-reg}, ta chọn mô hình gồm 8 biến theo phương pháp chọn tất cả dựa vào chuẩn BIC. Tuy nhiên, các hệ số hồi quy vẫn chưa đạt được mức ý nghĩa thống kê $0.05$ nên ta sẽ biến đổi log biến được giải thích $wage$ thành biến $lwage$, và xây dựng lại mô hình dựa trên 8 biến được chọn ra.

\begin{equation*}
	\begin{multlined}
		\texttt{lwage } = \beta_0 + \text{ } \beta_{educ}\times \texttt{educ} \text{ } + \text{ } \beta_{tenure} \times \texttt{tenure} \text{ }+\text{ }\beta_{female} \times \texttt{female} \text{ } \\
		+\text{ }\beta_{smsa} \times\texttt{smsa}\text{ }  
		+ \text{ }\beta_{west} \times \texttt{west} \text{ } 
		+ \text{ }\beta_{trade}\times \texttt{trade} + \text{ }\beta_{services} \times \texttt{services} \text{ }\\ + \text{ }\beta_{profocc}\times \texttt{profocc}
	\end{multlined}
\end{equation*}


\begin{figure}[H]
	\centering
	\includegraphics[width=0.7\textwidth]{../Photo Of Result/model-final-4}
	\caption{Mô hình hồi quy với 8 biến được chọn và đã qua chuẩn hóa log}
	\label{model-final-4}
\end{figure}

Dựa vào kết quả trong hình \ref{model-final-4}, ta được mô hình hồi quy:
\begin{equation*}
\begin{multlined}
	\texttt{lwage } = \text{ } 0.0493\times \texttt{educ} \text{ } + \text{ } 0.0169 \times \texttt{tenure} \text{ }-\text{ }0.2846 \times \texttt{female} \text{ } + 0.1319 \text{ } \times \texttt{smsa} \text{ } \\
	+\text{ }0.1124 \times\texttt{west}\text{ } -\text{ }0.2689\times \texttt{trade} -\text{ }0.2684 \times \texttt{services} \text{ } + \text{ }0.2509\times \texttt{profocc} + 0.951
\end{multlined}
\end{equation*}

\begin{figure}[H]
	\centering
	\includegraphics[width=\textwidth]{../Photo Of Result/vif-4}	
	\caption{Hệ số VIF của mô các biến được chọn}
	\label{vif}
\end{figure}

Từ kết quả chạy từ R trong hình \ref{vif}, ta thấy được mô hình không vi phạm điều kiện nào của mô hình tuyến tính. Trong mô hình không tồn tại hiện tượng đa cộng tuyến (VIF < 5.0).

\begin{figure}[H]
	\includegraphics[width=\textwidth]{../Photo Of Result/diagnostic-plot-4}
	\caption{Các biểu đồ đánh giá mô hình}
	\label{diagnostic}
\end{figure}

Dựa vào hình \ref{diagnostic} (Noraml Q-Q), ta có thể thấy được các điễm nằm gần như trên đường chéo, tức là sai số có phương sai không đôi và đã tuân theo phân phối chuẩn.

\subsection*{Kết luận}
Khi kiểm tra các giả thiết ý nghĩa của mô hình:
\begin{itemize}
	\item Vấn đề đa cộng tuyến trong bộ dữ liệu xảy ra rất nhiều, nhưng khi sử dụng các phương pháp chọn biến thì đã vô tình loại bỏ được hiện tượng đa cộng tuyến.
	\item Phần dư $\epsilon$ trong mô hình chọn tuân theo phân phối chuẩn.
\end{itemize}
Mô hình được chọn có ý nghĩa thống kê gồm 8 biến với hệ số xác định hiệu chỉnh $0.4849$, tức là mô hình được lựa chọn chỉ giải thích được khoảng $49\%$ bộ dữ liệu. Ta có thấy được mức lương được ảnh hưởng nhiều bởi biến giới tính, biến về vị trí; các biến về số năm đào tạo, số năm làm việc không ảnh hưởng gì nhiều đến mức lương.

Tuy nhiên, việc chọn mô hình này không hiệu quả vì chỉ giải thích được dưới $50\%$ bộ dữ liệu này. Với bộ dữ liệu này, nên cân nhắc một phương pháp mới phức tạp hơn để dự đoán giá lương chứ không đơn thuần chỉ sử dụng mô hình hồi quy tuyến tính.

\end{document}