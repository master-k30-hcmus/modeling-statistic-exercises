\section{Dữ liệu 1: Mô hình hồi quy đa biến}

\subsection*{Giới thiệu bộ dữ liệu}
Bộ dữ liệu được tìm thấy trên trang Kaggle - một cộng đồng trực tuyến về khoa học dữ liệu và học máy. Đó là bộ dữ liệu \textbf{Chi phí Y tế Cá nhân} \footnote{\url{https://www.kaggle.com/mirichoi0218/insurance}} (\textit{Medical Cost Personal Datasets}). Đây là một bộ dữ liệu được lấy ra từ cuốn \textit{Machine Learning with R} của Brett Lantz, một cuốn sách giới thiệu về học máy bằng R.

Bộ dữ liệu ghi lại các thông tin về thông tin của người đăng kí bảo hiểm và chi phí mà bảo hiểm y tế phải chi trả cho cá nhân đó. Bộ dữ liệu có 1338 quan trắc, gồm 7 biến sau:
\begin{enumerate}
	\item \texttt{age}: tuổi
	\item \texttt{sex}: giới tính
	\item \texttt{bmi}: chỉ số đo cân nặng, sử dụng tỷ lệ giữa cân nặng và chiều cao ($kg / m$), chỉ số BMI lý tưởng là từ 18.5 đến 24.9.
	\item \texttt{children}: số lượng trẻ em được bao gồm trong bảo hiểm y tế của người đăng kí.	
	\item \texttt{smoker}: 1 nếu người đó có hút thuốc, ngược lại là 0.
	\item \texttt{region}: vùng miền ở US, bao gồm Đông Bắc (\textit{northeast}), Đông Nam (\textit{southeast}), Tây Nam (\textit{southwest}), Tây Bắc (\textit{northwest}).
	\item \texttt{charges}: Chi phí y tế của cá nhân được chi trả bởi bảo hiểm y tế.
\end{enumerate}

Nhận thấy biến \texttt{region} có bốn giá trị, để thuận tiện cho việc hồi quy mô hình đa biến, chúng ta cần phải tách \texttt{region} thành ba biến giả lần lượt là \texttt{region\_ne} - vùng Đông Bắc, \texttt{region\_se} - vùng Đông Nam và \texttt{region\_sw} - vùng Tây Nam, nếu không nằm trong 3 vùng này thì nó là vùng Tây Bắc. Vậy bộ dữ liệu hiện tại có tất cả 9 biến.

Kiểm tra sự trùng lặp dữ liệu trong bộ dữ liệu, ta có kết quả từ phầm mềm R ở hình \ref{fig-a1:dataset-duplicated}, thấy rằng chỉ tồn tại một dữ liệu bị trùng, ta tiến hành loại bỏ dữ liệu này. Vậy bộ dữ liệu hiện tại có 1337 quan trắc.
\begin{figure}[H]
	\centering
	\includegraphics[width=0.7\linewidth]{images/A1/dataset-duplicated}
	\caption{Dữ liệu bị trùng lặp}
	\label{fig-a1:dataset-duplicated}
\end{figure}

Một vài quan trắc đầu tiên trong bộ dữ liệu được thể hiện trong hình \ref{fig-a1:head-dataset} và số~chiều của nó: 1337 dòng (quan trắc) và 9 cột (biến).
\begin{figure}[H]
	\centering
	\includegraphics[width=0.9\linewidth]{images/A1/head-dataset}
	\caption{Một vài quan trắc đầu tiên và số chiều của bộ dữ liệu}
	\label{fig-a1:head-dataset}
\end{figure}

Phân bố của 7 biến ban đầu ở hình \ref{fig-a1:plot-vars} và trung bình tổng của từng biến theo biến phụ thuộc \texttt{charges} ở hình \ref{fig-a1:boxplot-vars}.
\begin{figure}[H]
	\centering
	\includegraphics[width=1\linewidth]{images/A1/plot-vars}
	\caption{Phân bố của 7 biến ban đầu}
	\label{fig-a1:plot-vars}
\end{figure}

\begin{figure}[H]
	\centering
	\includegraphics[width=1\linewidth]{images/A1/boxplot-vars-by-charges}
	\caption{Biểu đồ boxplot của 6 biến so đối với biến phụ thuộc \texttt{charges}}
	\label{fig-a1:boxplot-vars}
\end{figure}

Một số điều thú vị quan sát được ở hai hình \ref{fig-a1:plot-vars} và \ref{fig-a1:boxplot-vars}, ta xét lần lượt từng biến:
\begin{itemize}
	\item \texttt{sex}: dù là nam hay nữ thì phân bố giữa hai giới này đều xấp xỉ nhau, đồng thời chi phí trung bình mà bảo hiểm y tế chi trả cũng xấp xỉ nhau.
	\item \texttt{smoker}: số lượng người hút thuốc ít hơn số lượng người không hút thuốc có trong bộ dữ liệu, nhưng chi phí trung bình mà bảo hiểm y tế chi trả cho nhóm này thì hoàn toàn cao hơn rất nhiều, điều này khá hiển nhiên.
	\item \texttt{region}: phân bố của các vùng và chi phí trung bình mà bảo hiểm y tế chi trả ở từng vùng cũng đều xấp xỉ nhau. 
	\item \texttt{bmi}: chỉ số BMI có phân bố dạng chuẩn, và chi phí trung bình mà bảo hiểm y tế chi trả cũng tăng dần đều theo chỉ số này, điều này cũng hợp lý vì khi chỉ số BMI càng cao thì khả năng bị béo phì cũng tăng. 
	\item \texttt{age}: tuổi tác có phân bố ngẫu nhiên, và chi phí trung bình mà bảo hiểm y tế chi~trả cũng tăng dần theo tuổi, điều này cũng khá hiển nhiên.
	\item \texttt{children}: phân bố của trẻ em được hưởng theo bảo hiểm bị lệch hẳn về bên trái, nên chi phí trung bình mà bảo hiểm y tế chi trả cho 4-5 trẻ em có thể bị sai lệch do mất cân bằng dữ liệu.
\end{itemize}

Nhận xét tổng quan, ta thấy rằng chi phí bảo hiểm y tế chi trả \texttt{charges} có khả năng phụ thuộc vào các đặc tính như người hút thuốc \texttt{smoker}, chỉ số \texttt{bmi}, tuổi tác \texttt{age} và số trẻ em phụ thuộc \texttt{children}. Các đặc tính còn lại như vùng miền \texttt{region} và giới tính \texttt{sex} có thể sẽ không ảnh hưởng nhiều đến \texttt{charges}.

\subsection*{Phân tích và chọn mô hình}

Xét mô hình đầy đủ sau:
\begin{equation}\label{a1-model-full}
	\begin{split}
		\texttt{charges} = \beta_0 + &\beta_1 \times \texttt{age} + \beta_2 \times \texttt{sex} + \beta_3 \times \texttt{bmi} + \beta_4 \times \texttt{children} + \beta_5 \times \texttt{smoker}\\ + &\beta_6 \times \texttt{region\_ne} + \beta_7 \times \texttt{region\_se} + \beta_8 \times \texttt{region\_sw} + \epsilon
	\end{split}
\end{equation}

Mô hình hồi quy đầy đủ có các thông số ở hình \ref{fig-a1:model-full}, đúng như dự đoán, biến vùng~miền \texttt{region} và giới tính \texttt{sex} không có ý nghĩa thống kê, và các biến còn lại có ý nghĩa thống kê tương đối cao. Mặt khác, đặc tính hút thuốc có hệ số cao nhất trong số còn lại, với một người hút thuốc, họ phải chi trả khoảng 24,000 USD.
\begin{figure}[H]
	\centering
	\includegraphics[width=0.7\linewidth]{images/A1/model-full}
	\caption{Mô hình đầy đủ}
	\label{fig-a1:model-full}
\end{figure}

Tiến hành sử dụng phương pháp tính hệ số VIF để kiểm tra hiện tượng đa cộng~tuyến có trong mô hình này, kết quả từ phần mềm R ở hình \ref{fig-a1:model-full-vif} cho thấy các hệ số VIF đều dưới 5, chứng tỏ không tồn tại hiện tượng này trong mô hình.
\begin{figure}[H]
	\centering
	\includegraphics[width=0.5\linewidth]{images/A1/model-full-vif}
	\caption{Hiện tượng đa cộng tuyến trong mô hình đầy đủ}
	\label{fig-a1:model-full-vif}
\end{figure}

Mô hình giải thích được khoảng 75\% sự thay đổi thay đổi của chi phí chi trả y tế \texttt{charges} dựa trên các đặc tính như hút thuốc, tuổi tác, chỉ số BMI và số lượng trẻ em phụ thuộc. Tuy nhiên, dựa vào hình \ref{fig-a1:model-full-plot}, có thể thấy ở biểu đồ phần dư có kì vọng có vẻ gần bằng 0, nhưng phương sai không phải hằng số.
\begin{figure}[H]
	\centering
	\includegraphics[width=0.8\linewidth]{images/A1/model-full-plot}
	\caption{Các biểu đồ của mô hình lựa chọn}
	\label{fig-a1:model-full-plot}
\end{figure}

Ta tiến hành chọn mô hình, hi vọng các biến được chọn sẽ cải thiện tình trạng của mô hình đầy đủ. Do mô hình đầy đủ đã có nhiều biến có ý nghĩa thống kê, nhóm em sử~dụng phương pháp Stepwise lùi để chọn mô hình phù hợp nhanh hơn, và tiêu chuẩn BIC cho mô hình đơn giản có 9 biến trên. Các bước và kết quả chọn mô hình, các thông~số của nó được thể hiện ở hình \ref{fig-a1:model-bic}.
\begin{figure}[H]
	\centering
	\subfloat[Các bước và kết quả chọn mô hình]
	{\includegraphics[width=.5\linewidth]{images/A1/model-bic-anova}}\hfill
	\subfloat[Mô hình lựa chọn]
	{\includegraphics[width=.5\linewidth]{images/A1/model-bic}}
	\caption{Mô hình lựa chọn với phương pháp Stepwise và tiêu chuẩn BIC}
	\label{fig-a1:model-bic}
\end{figure}

Đồng thời mô hình cũng không có hiện tượng đa cộng tuyến giữa các biến được lựa~chọn.
\begin{figure}[H]
	\centering
	\includegraphics[width=0.35\linewidth]{images/A1/model-bic-vif}
	\caption{Hiện tượng đa cộng tuyến trong mô hình lựa chọn}
	\label{fig-a1:model-bic-vif}
\end{figure}

Tuy nhiên, các biến được lựa chọn khá tương đồng với những biến có ý nghĩa với mô hình đầy đủ, hình \ref{fig-a1:model-bic-plot} cho thấy các biểu đồ trong mô hình lựa chọn không khác mấy với mô hình đầy đủ. 
\begin{figure}[H]
	\centering
	\includegraphics[width=0.7\linewidth]{images/A1/model-bic-plot}
	\caption{Các biểu đồ của mô hình lựa chọn}
	\label{fig-a1:model-bic-plot}
\end{figure}

Việc chuẩn hóa dữ liệu (hình \ref{fig-a1:scaled-model-full}) hay xóa bỏ các dữ liệu ngoại lai (\ref{fig-a1:new-model-full}) cũng không thay đổi được gì.
\begin{figure}[H]
	\centering
	\subfloat[Thông số mô hình]
	{\includegraphics[width=0.5\linewidth]{images/A1/scaled-model-full}}\hfill
	\subfloat[Các biểu đồ của mô hình]
	{\includegraphics[width=0.5\linewidth]{images/A1/plot-scaled-model}}
	\caption{Mô hình lựa chọn với bộ dữ liệu chuẩn hóa}
	\label{fig-a1:scaled-model-full}
\end{figure}

\begin{figure}[H]
	\centering
	\subfloat[Thông số mô hình]
	{\includegraphics[width=0.5\linewidth]{images/A1/new-model}}\hfill
	\subfloat[Các biểu đồ của mô hình]
	{\includegraphics[width=0.5\linewidth]{images/A1/plot-new-model}}
	\caption{Mô hình lựa chọn với bộ dữ liệu được xóa ngoại lai}
	\label{fig-a1:new-model-full}
\end{figure}

\subsection*{Nhận xét và kết luận}

Khi kiểm tra các điều kiện ý nghĩa của mô hình:
\begin{itemize}
	\item Vấn đề đa cộng tuyến các mô hình lựa chọn đều được đảm bảo là không xảy ra.
	\item Tuy nhiên, phần dư $\epsilon$ trong các mô hình đều không tuân theo phân phối chuẩn, kỳ vọng có thể bằng 0, nhưng phương sai không là một hằng số.
\end{itemize}

Vậy các \textbf{mô hình không có ý nghĩa}, dù đã xét các vấn đề về dữ liệu là ngoại lai và chuẩn hóa, nhưng cũng không thay đổi được mô hình, có thể mô hình tuyến tính chưa phù hợp để thể giải thích tốt cho chi phí chi trả y tế \texttt{charges}. Mô hình tuyến tính có thể giải thích được khoảng 75\% sự biến thiên của \texttt{charges}, phụ thuộc nhiều nhất là khi người có hút thuốc hay không. Vì vậy, mô hình vẫn có thể sử dụng để dự đoán, nhưng cần phải cẩn thận và xem xét về kết quả.

Những điều có thể thử để cải thiện hiệu quả mô hình:
\begin{itemize}
	\item Cần sử dụng các mô hình hồi quy phi tuyến khác như Support Vector Regression, Random Forest Regression, Decision Tree Regression,... để có những bằng chứng chắc chắn hơn về ý nghĩa của mô hình.
	\item Biến đổi một số biến độc lập sử dụng để xây dựng mô hình: bình phương, $\log$, box-cox,...
	\item Cần bổ sung thêm các đặc tính khác liên quan đến chi phí chi trả y tế như là có uống rượu, tiền sử bệnh (tiểu đường, huyết áp,...) để có thể mô tả biến phụ thuộc tốt hơn.
\end{itemize}
